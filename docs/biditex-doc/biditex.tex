\documentclass[twocolumn]{article}
%\documentclass{article}
\usepackage[cp1255]{inputenc}
\usepackage[english,hebrew]{babel}

%%%%%%%%%%%%%%%%%%%%%%%%%%
% Ivritex
\usepackage{hebfont}
%%%%%%%%%%%%%%%%%%%%%%%%%%
% Nikud
%\usepackage{culmus}
%\HeblatexRedefineL
%%%%%%%%%%%%%%%%%%%%%%%%%%

\usepackage{newcent}
\global\emergencystretch = .8\hsize
%%%%%%%%%%%%%%%%%%%%%%%%%%%%%%%%%%%%%%%%%%%%%%%%%%%%%%%%%%%%%
% Some basic macros for me
\newcommand{\BiDiTeX}{\L{BiDi\TeX{}}}
\newcommand{\bs}{$\backslash$}
\newcommand{\tagL}{\texttt{\L{\bs L\{\ldots\}}}}
\newcommand{\tagR}{\texttt{\L{\bs R\{\ldots\}}}}
\newcommand{\bidion}{\texttt{\L{\%BIDION}}}
\newcommand{\bidiltr}{\texttt{\L{\%BIDILTR}}}
\newcommand{\bidioff}{\texttt{\L{\%BIDIOFF}}}
\newcommand{\biditag}{\texttt{\L{\%BIDITAG}}}
\newcommand{\minus}{\L{-}}
\newcommand{\code}[1]{\L{\begin{verse}\texttt{#1}\end{verse}}}
\newcommand{\exframe}[1]{\begin{quote} #1 \end{quote} }

%%%%%%%%%%%%%%%%%%%%%%%%%%%%%%%%%%%%%%%%%%%%%%%%%%%%%%%

%BIDITAG code
%BIDITAG label
%BIDITAG ref

%%%%%%%%%%%%%%%%%%%%%%%%%%%%%%%%%%%%%%%%%%%%%%%%%%%%%%%

%BIDION
\title{\BiDiTeX{} -- כיווניות למסמכי \LaTeX{} בעברית}
\author{ארתיום בייליס (טונקיך)}
\begin{document}
\maketitle
\begin{abstract}
זהו מסמך תיעוד לשימוש ב-\BiDiTeX{} -- תוסף ל-\LaTeX, שמאפשר תמיכה אוטומטית בכיווניות במסמכים בעברית. הוא נועד לעבוד יחד עם חבילת ivritex -- החבילה שמוסיפה תמיכה בעברית ב-\LaTeX. מסמך זה נבנה תוך שימוש ב-\BiDiTeX{} ומהווה הדגמה של יכולותיו. הוא מופץ תחת רישיון ייחוס-שיתוף זהה 1.0 ישראל של Creative Commons ו-GFDL.
\end{abstract}

%\tableofcontents


\section{תמיכה בעברית ב-\LaTeX{} -- מה הבעיה?}
חבילת ivritex מספקת תמיכה בעברית במסמכי \LaTeX. יחד עם זו, היא לא פותרת את אחת הבעיות החשובות ביותר של שימוש בעברית -- תמיכה בכיווניות. הסתכלו בפסקה הבאה:

\exframe{
שמי ארתיום (Artyom), נולדתי ב-1981 בקייב (אוקראינה).
}

ללא התייחסות מיוחדת לכיווניות, היא תיראה כך:
%BIDIOFF

\exframe{
שמי ארתיום (Artyom), נולדתי ב-1981 בקייב (אוקראינה).
}

%BIDION

על מנת לגרום לפסקה הזו להיראות כראוי יש להוסיף תגי כיווניות  בצורה מפורשת ולהפוך סוגריים.

\exframe{
שמי ארתיום )\bs L\{Artyom\}(, נולדתי ב-\bs L\{1981\} בקייב )אוקראינה(.
}

כמובן "כתיבה" כזו עלולה להפוך לסיוט בגלל ערבוב של תגים באנגלית ועברית שהופכת את המלל לבלתי קריא לחלוטין. כמו כן, צורת כתיבה כזו לא מאפשרת ייבוא טקסטים רגילים בצורה פשוטה (ע"י העתק-והדבק). אותה בעיה קיימת בהוספת מילים או משפטים בעברית לטקסטים בשפות שכתובות משמאל לימין. כל משפט חייב להיות מוגדר ע"י תג מיוחד.

חבילת \BiDiTeX{} נועדה לפתור את הבעיה -- לבצע ניתוח טקסט לפני מסירתו ל-\LaTeX{} והוספת התגים הנדרשים על-פי אלגוריתם שהוגדר בתקן UNICODE.


\section{שימוש ב-\BiDiTeX.}
השימוש ב-\BiDiTeX{} הוא פשוט. כותבים קובץ \LaTeX{} רגיל, מסביב לכל קטע שאמור להיות מטופל ע"י \BiDiTeX{} מוסיפים תגים מיוחדים \bidion{} להפעלה ותג \bidioff{} לכיבוי -- כל אחד בשורה נפרדת. בכל הטקסט שנמצא בין התגים האלו יתווספו תגי כיווניות לפי הצורך.  ראה איור  \ref{example} לדוגמה.
\begin{figure}[h]
\caption{דוגמה לקוד של מסמך פשוט.\label{example}}
%BIDIOFF
\unsethebrew
\texttt{\bs documentclass\{article\}\\
\bs usepackage[cp1255]\{inputenc\}\\
\bs usepackage[english,hebrew]\{babel\}\\
\bidion\\
\bs begin\{document\}\\
\sethebrew
\sethebrew
%BIDION
"שלום עולם" זה Hello World באנגלית.\\
%BIDIOFF
\unsethebrew
\bs end\{document\}\\
\bidioff}
\sethebrew
%BIDION
\end{figure}

כאשר עובדים באנגלית ומשלבים בתוכו עברית, ניתן להשתמש בתג \bidiltr{} שנועד לטפל במקרה זה. ראה איור \ref{ltr}.
\begin{figure}[h]
\caption{שימוש בתג \bidiltr.\label{ltr}}
%BIDILTR
\unsethebrew
\texttt{\bidiltr\\
The meaning of word "שלום" is peace.\\
\bidioff}
\sethebrew
%BIDION
\end{figure}

למרות ששימוש בתגים האלו הוא שקוף מבחינת המשתמש ממולץ להכיר כיצד \BiDiTeX{} מטפל במלל ומוסיף תגים.

\subsection{טיפול בתגי \LaTeX.}
אחת הבעיות ה"קשות" עבור \BiDiTeX{} היא להבחין -- מהו הטקסט הרגיל שיש לטפל בו ולהוסיף תגי כיווניות ומתי אסור להכניס תגים אלו. למעשה נדרשת יכולת אבחנה פשוטה בין תוכן המסמן לבין תגי העיצוב שלו. מכיוון שב-\LaTeX{} קיים מגוון מאוד רחב של תגים שיכולים להתווסף ע"י חבילות שונות, אין זה אפשרי לגרום ל-\BiDiTeX{} להכיר את כולם. לכן, נבחרה קבוצה קטנה של סוגי התגים שנמצאים בשימוש הרחב ביותר.
\subsubsection{סוגי התגים.}
\begin{description}
\item[תגים עם טקסט] -- 
תגים נראים בצורה הבאה:
\code{\bs tag[\emph{text 1}]\ldots\{\emph{text 2}\}\\ \bs tag*[\emph{text 1}]\ldots\{\emph{text 2}\}}

כאשר \BiDiTeX{} חייב לטפל בטקסט שנמצא בין הסוגריים ולהוסיף תגי כיווניות במידת הצורך.
\item[תגים עם פקודות] --
תגים שנראים באותה צורה כמו תגים עם טקסט, רק שאסור לטפל בתוכן שנמצא בין הסוגריים על מנת לא לפגוע בתוכן שלהן, לדוגמה:
\code{\bs begin\{tabular\}\{|l|r|\}}
בשלב זה, \BiDiTeX{} מכיר רק את התגים הבאים כפקודות מסוג זה -- begin, end, L, R. תמיד ניתן להוסיף עוד תגים כאלה ע"י שימוש בפקודה \biditag. לדוגמה אפשר לגרום ל-\BiDiTeX{} להתעלם מתוכן התג "ref" ע"י הוספת השורה הבאה:
\code{\biditag{} ref}

\item[נוסחאות] --
נוסחאות שמשולבות בתוך מלל כמו: 
$e=\sum_{i=0}^{\infty}\frac{1}{i!}$
 שהוגדרו בין סימני "\$". \BiDiTeX{} מתעלם מהתוכן שלהן לחלוטין. יש לשים לב, שהנוסחה חייבה להיות כתובה בשורת טקסט אחת.

\unsethebrew
\verb+$e=\sum_{i=0}^{\infty}\frac{1}{i!}$+
\sethebrew

ולא פרוסה על מספר שורות:
\unsethebrew
%BIDIOFF
\begin{verbatim}
$e=\sum_{i=0}^{\infty}
    \frac{1}{i!}$
\end{verbatim}
%BIDION
\sethebrew
במקרים שזה בלתי אפשרי לכתוב נוסחה בשורה אחת, או הנוסחה מוגדרת בתוך סביבה כמו equation, יש לבטל את \BiDiTeX{} ע"י שימוש בתג "\bidioff".
\end{description}
\subsubsection{מגבלות חשובות}
\begin{itemize}
\item
כל טיפול בתגים וכיווניות מתבצע שורה-שורה. לכן, יש סיכון, שהפקודה או תג \LaTeX{} שיתפרס על יותר משורה אחת, יטופל בצורה לא נכונה.
\item
שילוב של תגי כיווניות \tagL{} ו-\tagR{} עלול לפגוע בסדר הסוגריים והתגים שהוצבו. לכן, במקרים מסובכים של פקודות מכוננות שהכיווניות עלולה להשתנות באחד הצדדים יש לטפל בבעיית כיווניות באופן מפורש.

ייתכן ובעתיד הבעיה תטופל בצורה שלא תפגע בדקדוק של \LaTeX.


\end{itemize}
\subsection{הפיכת סוגריים.}
\BiDiTeX{}
תומך בהפיכת סוגריים על-פי הדרישות של תקן UNICODE. הוא מכיר את כל סוגי הסוגריים שנמצאים בקידוד ASCII -- <>, (), [], \{,\} (שמיוצגים ע"י \{\bs{} ו-\}\bs). אין תמיכה בהפיכת סוגריים אחרים, שאמורים להיות מטופלים לפי התקן, למשל -- סימן שייכות "$\in$".


\subsection{טיפול במקף.}
חבילת ivritex מחליפה סימן מינוס "\minus" בסימן מקף-עליון "-". אבל זה מתבצע גם עבור סידרה של שנים או שלושה מינוסים\footnote{הבעיה \emph{לא קיימת} כשמשתמשים בחבילת ניקוד -- nikud.berlios.de ומשלבים סגנון/חבילה culmus. יש לשים לב, שהפתרון המוצע הוא חלקי, בגלל שהגופן המשמש להצגת סימן "--" אינו עברי, לכן המראה של הסימן לא טבעי כשמשתמשים בחבילת ivritex כפי-שהיא.}
 "\minus\minus"
 או "\minus\minus\minus", שלפי כללי \LaTeX{} אמורים להפוך למקף "--" או "---" (קצר או ארוך יותר). זה מקשה על הוספת סימנים אלו לטקסט. למרות זאת, \BiDiTeX{} מטפל בשלב זה במקרים האלו והופך את הסימון "\minus\minus" ל-"--" ואת הסימון "\minus\minus\minus" ל-"---"; כאשר סימן מינוס רגיל "\minus" עדיין הופך למקף-עליון "-".

מכיוון שהפתרון הזה אינו פתרון אמתי אלא מעקף, החלפת סימני מינוס במקף לא מתבצעת כברירת מחדל אלא מופעלת ע"י מסירת פרמטר "\L{-m}" ל-biditex.

\section{התקנה והפעלה.}
\subsection{התקנה.}
הוראות התקנה ניתן למצוא בקובץ README שבא יחד עם קוד המקור. הקוד נבדק על שתי פלטפורמות: Debian Etch ו-Ubuntu Dapper יחד עם ivritex 1.2.1-1 ו-nikud, אך צפוי לעבוד בכל פלטפורמה שניתן להשתמש בה ב-libfribidi.
\subsection{הפעלה.}

כפי שהוסבר לעיל, יש להריץ \BiDiTeX{} לפני הפעלת \LaTeX{}. לדוגמה:

\code{biditex input.tex -o tmp.tex}

רק אז אפשר להפעיל את \LaTeX{} על קובץ tmp.tex. 
\BiDiTeX{}
מאפשר לקבוע אחד משלושת הקידודים הבאים: utf8, cp1255, iso8859-8 ע"י מסירת פרמטר מתאים, לדוגמה:

\code{biditex -e cp1255 test.tex -o result.tex}

קידוד ברירת המחדל הוא utf8.
\subsection{דוגמאות לשימוש.}
מסמך זה, יחד עם המקור שלו מהווים הדגמה מקיפה לשימוש ב-\BiDiTeX, יחס עם זו, הוא מאוד מסובך להתחלה מכיוון שמשתמש בהרבה כלים שקיימים ב-\LaTeX{} ומשלב כמות גדולה של טקסטים בעברית ואנגלית. מסמכים לדגומה פשוטים יתווספו בהמשך.
\end{document}
%BIDIOFF
