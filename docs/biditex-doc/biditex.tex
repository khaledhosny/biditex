%\documentclass[twocolumn]{article}
\documentclass{article}
\usepackage[cp1255]{inputenc}
\usepackage[english,hebrew]{babel}

%%%%%%%%%%%%%%%%%%%%%%%%%%
% Ivritex
\usepackage{hebfont}
\usepackage{biditex}
%%%%%%%%%%%%%%%%%%%%%%%%%%
% Nikud
\usepackage{culmus}
%\HeblatexRedefineL
%%%%%%%%%%%%%%%%%%%%%%%%%%

%\usepackage{newcent}
\global\emergencystretch = .8\hsize
%%%%%%%%%%%%%%%%%%%%%%%%%%%%%%%%%%%%%%%%%%%%%%%%%%%%%%%%%%%%%
% Some basic macros for me
\newcommand{\BiDiTeX}{\L{BiDi\TeX{}}}
\newcommand{\bs}{$\backslash$}
\newcommand{\tagL}{\texttt{\L{\bs L\{\ldots\}}}}
\newcommand{\tagR}{\texttt{\L{\bs R\{\ldots\}}}}
\newcommand{\bidion}{\texttt{\L{\%BIDION}}}
\newcommand{\bidiltr}{\texttt{\L{\%BIDILTR}}}
\newcommand{\bidioff}{\texttt{\L{\%BIDIOFF}}}
\newcommand{\biditag}{\texttt{\L{\%BIDITAG}}}
\newcommand{\minus}{\L{-}}
\newcommand{\code}[1]{\L{\begin{verse}\texttt{#1}\end{verse}}}
\newcommand{\exframe}[1]{\begin{quote} #1 \end{quote} }

%%%%%%%%%%%%%%%%%%%%%%%%%%%%%%%%%%%%%%%%%%%%%%%%%%%%%%%

%BIDITAG code
%BIDITAG label
%BIDITAG ref
%BIDIDICTAG בידיטך BiDiTeX
%BIDIDICTAG קו bs


%%%%%%%%%%%%%%%%%%%%%%%%%%%%%%%%%%%%%%%%%%%%%%%%%%%%%%%

%BIDION
\כותרת{\BiDiTeX{} -- כיווניות למסמכי \LaTeX{} בעברית}
\מחבר{ארתיום בייליס (טונקיך)}
\התחל{מסמך}
\צורכותרת
\התחל{תקציר}
מסמך זה תיעוד לשימוש ב-\BiDiTeX{} -- תוסף ל-\LaTeX, שמספק תמיכה אוטומטית בכיווניות למסמכים בעברית. הוא תוכנן לעבוד יחד עם חבילת ivritex -- החבילה שמוסיפה תמיכה בעברית ל-\LaTeX.

מסמך זה נבנה תוך שימוש ב-\BiDiTeX{} ומהווה הדגמה של יכולותיו. הוא מופץ תחת רישיון ייחוס-שיתוף זהה 1.0 ישראל של Creative Commons ו-GFDL.
\סיים{תקציר}

\tableofcontents
\listoffigures
\listoftables


\סעיף{תמיכה בעברית ב-\LaTeX{} -- מה הבעיה?}
חבילת ivritex מספקת תמיכה בעברית במסמכי \LaTeX. יחד עם זו, היא לא פותרת את אחת הבעיות החשובות ביותר של שימוש בעברית -- תמיכה בכיווניות. הסתכלו בפסקה הבאה:

\exframe{
שמי ארתיום (Artyom), נולדתי ב-1981 בקייב (אוקראינה).
}

ללא התייחסות מיוחדת לכיווניות, היא תיראה כך:
%BIDIOFF

\exframe{
שמי ארתיום (Artyom), נולדתי ב-1981 בקייב (אוקראינה).
}

%BIDION

על מנת לגרום לפסקה הזו להיראות כראוי יש להוסיף תגי כיווניות  בצורה מפורשת ולהפוך סוגריים.

\exframe{
שמי ארתיום )\bs L\{Artyom\}(, נולדתי ב-\bs L\{1981\} בקייב )אוקראינה(.
}

כמובן "כתיבה" כזו עלולה להפוך לסיוט בגלל ערבוב של תגים באנגלית ועברית שהופכת את המלל לבלתי קריא לחלוטין. כמו כן, צורת כתיבה כזו לא מאפשרת ייבוא טקסטים רגילים בצורה פשוטה (ע"י העתק-והדבק). אותה בעיה קיימת בהוספת מילים או משפטים בעברית לטקסטים בשפות שכתובות משמאל לימין. כל משפט חייב להיות מוגדר ע"י תג מיוחד.

חבילת \BiDiTeX{} נועדה לפתור את הבעיה -- לבצע ניתוח טקסט לפני מסירתו ל-\LaTeX{} והוספת התגים הנדרשים על-פי אלגוריתם שהוגדר בתקן UNICODE.


\סעיף{שימוש ב-\BiDiTeX.}
השימוש ב-\BiDiTeX{} הוא פשוט. מוסיפים חבילה biditex ע"י הוספת הוראה:
\L{\bs usepackage\{biditex\}}.
כותבים קובץ \LaTeX{} רגיל, מסביב לכל קטע שאמור להיות מטופל ע"י \BiDiTeX{} מוסיפים תגים מיוחדים \bidion{} להפעלה ותג \bidioff{} לכיבוי -- כל אחד בשורה נפרדת. בכל הטקסט שנמצא בין התגים האלו יתווספו תגי כיווניות לפי הצורך.  ראה איור  \ראה{example} לדוגמה.
\begin{figure}[h]
\caption{דוגמה לקוד של מסמך פשוט.\label{example}}
%BIDIOFF
\unsethebrew
\texttt{\bs documentclass\{article\}\\
\bs usepackage[cp1255]\{inputenc\}\\
\bs usepackage[english,hebrew]\{babel\}\\
\bs usepackage\{biditex\}\\
\bidion\\
\bs begin\{document\}\\
\sethebrew
\sethebrew
%BIDION
"שלום עולם" זה Hello World באנגלית.\\
%BIDIOFF
\unsethebrew
\bs end\{document\}\\
\bidioff}
\sethebrew
%BIDION
\end{figure}

כאשר עובדים באנגלית ומשלבים בתוכו עברית, ניתן להשתמש בתג \bidiltr{} שנועד לטפל במקרה זה. ראה איור \ראה{ltr}.
\begin{figure}[h]
\caption{שימוש בתג \bidiltr.\label{ltr}}
%BIDILTR
\unsethebrew
\texttt{\bidiltr\\
The meaning of word "שלום" is peace.\\
\bidioff}
\sethebrew
%BIDION
\end{figure}

בנוסף ניתן להשתמש בתגים בעברית, לדוגמה אפשר לכתוב "\bs הדגש\{טקסט\}" במקום להשתמש בתג "\bs emph\{...\}". למרות ששימוש בתגים האלו הוא שקוף מבחינת המשתמש ממולץ להכיר כיצד \BiDiTeX{} מטפל במלל ומוסיף תגים.

\תתסעיף{תרגום התגים.}
\בידיטך{} מאפשר להשתמש בתגים בעברית ומתרגם אותם למקביליהם באנגלית, לדוגמה: "הדגש" הופך ל-"emph", "התחל" מתורגם ל-"begin". כל תג בעברית חייב להתחיל כמו כל תג ב-\LaTeX{} בסימן "\bs", או אם מדובר בסביבה (כמו enumerate), הוא צריך להופיע בתוך הגדרות בסביבה (begin או end). לדוגמה: הטקסט שמוצג באיור \ראה{tagheb} יתורגם לטקסט שמוצג באיור \ראה{tageng}.
\begin{figure}[h]
\caption{תגים בעברית \label{tagheb}}
%BIDIOFF
\begin{verbatim}
\סעיף}צבעים{
    \התחל}מספור{
        \פריט ירוק
        \פריט כחול
    \סיים}מספור{
\תתסעיף}משהו אחר.{
\end{verbatim}
%BIDION
\end{figure}
\begin{figure}[h]
\caption{תרגום תגים לאנגלית \label{tageng}}
%BIDILTR
\unsethebrew
\texttt{\bs section\{צבעים\}\\
\bs begin\{enumerate\}\\
\bs item ירוק\\
\bs item כחול\\
\bs end \{enumerate\}\\
\bs subsection\{ משהו אחר.\}}
\sethebrew
%BIDION
\end{figure}

תמיכה בתגים באנגלית מקלה על עובדה עם טקסט דו-כיווני ומגדילה קריאות של טקסט.

ניתן להוסיף תגים חדשים ע"י שימוש בפקודה \texttt{BIDIDICTAG\%} עבור תגים ו-\texttt{BIDIDICENV\%} עבור סביבות. לדוגמה, פקודה
\texttt{\L{\%BIDIDICTAG \R{התגשלי} mytag}}
מגדירה את בתג "\bs התגשלי" כ-"mytag\bs". את הרשימה המלאה של התגים והסביבות שמוגדרים כברירת מחדל ניתן למצוא בטבלאות \ראה{alltags} ו-\ראה{allenvs} בהתאם.


\begin{table}[p]
\caption{רשימת התגים המתורגמים\label{alltags}}
\center{
%BIDILTR
\unsethebrew
\begin{tabular}{|l|r|}
\hline
\R{תג באגנלית} & \R{תג בעברית} \\
\hline
L & ש \\
R & י \\
begin & התחל \\
end & סיים \\
title & כותרת \\
tableofcontents & תוכןעניינים \\
maketitle & צורכותרת \\
author & מחבר \\
date & תאריך \\
part & חלק \\
chapter & פרק \\
section & סעיף \\
subsection & תתסעיף \\
subsubsection & תתתתסעיף \\
emph & הדגש \\
item & פריט, נק \\
label & תג \\
ref & ראה \\
footnote & הערתשוליים \\
cite & צטט \\
ldots & נקודות \\

\hline
%BIDION
\end{tabular}
\sethebrew
}
\end{table}


\begin{table}[p]
\caption{רשימת הסביבות המתורגמות\label{allenvs}}
\center{
%BIDILTR
\unsethebrew
\begin{tabular}{|l|r|}
\hline
\R{תג באגנלית} & \R{תג בעברית} \\
\hline

\L{itemize} & פירוט \\
description & תיאור \\
enumerate & מספור \\
verse & שיר \\
document & מסמך \\
abstract & תקציר \\

\hline
%BIDION
\end{tabular}
\sethebrew
}
\end{table}



\תתסעיף{טיפול בתגי \LaTeX.}
אחת הבעיות ה"קשות" עבור \BiDiTeX{} היא להבחין -- מהו הטקסט הרגיל שיש לטפל בו ולהוסיף תגי כיווניות ומתי אסור להכניס תגים אלו. למעשה נדרשת יכולת אבחנה פשוטה בין תוכן המסמן לבין תגי העיצוב שלו. מכיוון שב-\LaTeX{} קיים מגוון מאוד רחב של תגים שיכולים להתווסף ע"י חבילות שונות, אין זה אפשרי לגרום ל-\BiDiTeX{} להכיר את כולם. לכן, נבחרה קבוצה קטנה של סוגי התגים שנמצאים בשימוש הרחב ביותר: begin, end, R, L, Lnum, ref, label, includegraphics, bibitem, cite.

\תתתתסעיף{סוגי התגים.}
\התחל{תיאור}
\פריט[תגים עם טקסט] -- תגים נראים בצורה הבאה:
\code{\bs tag[\emph{text 1}]\ldots\{\emph{text 2}\}\\ \bs tag*[\emph{text 1}]\ldots\{\emph{text 2}\}}

כאשר \BiDiTeX{} חייב לטפל בטקסט שנמצא בין הסוגריים ולהוסיף תגי כיווניות במידת הצורך.
\פריט[תגים עם פקודות] -- תגים שנראים באותה צורה כמו תגים עם טקסט, רק שאסור לטפל בתוכן שנמצא בין הסוגריים על מנת לא לפגוע בתוכן שלהן, לדוגמה:
\code{\bs begin\{tabular\}\{|l|r|\}}
בשלב זה, \BiDiTeX{} מכיר רק את התגים הבאים כפקודות מסוג זה -- begin, end, L, R. תמיד ניתן להוסיף עוד תגים כאלה ע"י שימוש בפקודה \biditag. לדוגמה אפשר לגרום ל-\BiDiTeX{} להתעלם מתוכן התג "ref" ע"י הוספת השורה הבאה:
\code{\biditag{} ref}

\פריט[נוסחאות] -- נוסחאות שמשולבות בתוך מלל כמו: 
$e=\sum_{i=0}^{\infty}\frac{1}{i!}$
 שהוגדרו בין סימני "\$" או בין סימני "\קו [" ו-"\קו ]". \BiDiTeX{} מתעלם מהתוכן שלהן לחלוטין. יש לשים לב, שהנוסחה חייבה להיות כתובה בשורת טקסט אחת.

\unsethebrew
\verb+$e=\sum_{i=0}^{\infty}\frac{1}{i!}$+
\sethebrew

ולא פרוסה על מספר שורות:
\unsethebrew
%BIDIOFF
\begin{verbatim}
$e=\sum_{i=0}^{\infty}
    \frac{1}{i!}$
\end{verbatim}
%BIDION
\sethebrew
במקרים שזה בלתי אפשרי לכתוב נוסחה בשורה אחת, או הנוסחה מוגדרת בתוך סביבה כמו equation, יש לבטל את \BiDiTeX{} ע"י שימוש בתג "\bidioff".
\סיים{תיאור}
\תתתתסעיף{מגבלות חשובות}
\התחל{פירוט}
\פריט כל טיפול בתגים וכיווניות מתבצע שורה-שורה. לכן, יש סיכון, שהפקודה או תג \LaTeX{} שיתפרס על יותר משורה אחת, יטופל בצורה לא נכונה.
\פריט שילוב של תגי כיווניות \tagL{} ו-\tagR{} עלול לפגוע בסדר הסוגריים והתגים שהוצבו. לכן, במקרים מסובכים של פקודות מכוננות שהכיווניות עלולה להשתנות באחד הצדדים יש לטפל בבעיית כיווניות באופן מפורש.

ייתכן ובעתיד הבעיה תטופל בצורה שלא תפגע בדקדוק של \LaTeX.
\סיים{פירוט}
\תתסעיף{הפיכת סוגריים.}
\בידיטך{} תומך בהפיכת סוגריים על-פי הדרישות של תקן UNICODE. הוא מכיר את כל סוגי הסוגריים שנמצאים בקידוד ASCII -- <>, (), [], \{,\} (שמיוצגים ע"י \{\bs{} ו-\}\bs). אין תמיכה בהפיכת סוגריים אחרים, שאמורים להיות מטופלים לפי התקן, למשל -- סימן שייכות "$\in$".


\תתסעיף{טיפול במקף.}
חבילת ivritex מחליפה סימן מינוס "\minus" בסימן מקף-עליון "-". אבל זה מתבצע גם עבור סידרה של שנים או שלושה מינוסים\footnote{הבעיה \הדגש{לא קיימת} כשמשתמשים בחבילת ניקוד -- nikud.berlios.de ומשלבים סגנון/חבילה culmus. יש לשים לב, שהפתרון המוצע הוא חלקי, בגלל שהגופן המשמש להצגת סימן "--" אינו עברי, לכן המראה של הסימן לא טבעי כשמשתמשים בחבילת ivritex כפי-שהיא.}
 "\minus\minus"
 או "\minus\minus\minus", שלפי כללי \LaTeX{} אמורים להפוך למקף "--" או "---" (קצר או ארוך יותר). זה מקשה על הוספת סימנים אלו לטקסט. למרות זאת, \BiDiTeX{} מטפל בשלב זה במקרים האלו והופך את הסימון "\minus\minus" ל-"--" ואת הסימון "\minus\minus\minus" ל-"---"; כאשר סימן מינוס רגיל "\minus" עדיין הופך למקף-עליון "-".

מכיוון שהפתרון הזה אינו פתרון אמתי אלא מעקף, החלפת סימני מינוס במקף לא מתבצעת כברירת מחדל אלא מופעלת ע"י מסירת פרמטר "\ש{-m}" ל-biditex.

מכיוון שגופן ברירת מחדל של \LaTeX{} לא נראה מתאים להצגת קו-מפריד בתוך הטקסט העברי, ניתן להגדיר גופן אחר עבור הצגת "קו-מפריד" או "טווח-מספרים" ע"י הוספת פקודה:

\code{\bs bididashfont\{fontname\}}

\סעיף{התקנה והפעלה.}
\תתסעיף{התקנה.}
הוראות התקנה ניתן למצוא בקובץ README שבא יחד עם קוד המקור. הקוד נבדק על שתי פלטפורמות: Debian Etch ו-Ubuntu Dapper יחד עם ivritex 1.2.1-1 ו-nikud, אך צפוי לעבוד בכל פלטפורמה שניתן להשתמש בה ב-libfribidi.
\תתסעיף{הפעלה.}

כפי שהוסבר לעיל, יש להריץ \BiDiTeX{} לפני הפעלת \LaTeX{}. לדוגמה:

\code{biditex input.tex -o tmp.tex}

רק אז אפשר להפעיל את \LaTeX{} על קובץ tmp.tex. פירוט מלא של פרמטרים ניתן למצוא בתיעוד man.

\תתסעיף{דוגמאות לשימוש.}
מסמך זה, יחד עם המקור שלו מהווים הדגמה מקיפה לשימוש ב-\BiDiTeX, יחס עם זו, הוא מאוד מסובך להתחלה מכיוון שמשתמש בהרבה כלים שקיימים ב-\LaTeX{} ומשלב כמות גדולה של טקסטים בעברית ואנגלית. התקנה של \BiDiTeX{} מספקת מסמך (example.tex) שמדגים שימוש בסיסי ב-\BiDiTeX.
\סיים{מסמך}
%BIDIOFF
