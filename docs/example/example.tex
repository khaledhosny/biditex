\documentclass{article}

\usepackage[cp1255]{inputenc}

%%%%%%%%%%%%%%%%%%%%%%%%%%%%%%%%%%%%%%%%%%%%%%%%%
% You can use both cp1255 and utf8 encoding     %
% however for utf8 you need include ucs package %
%%%%%%%%%%%%%%%%%%%%%%%%%%%%%%%%%%%%%%%%%%%%%%%%%

\usepackage[english,hebrew]{babel}

%%%%%%%%%%%%%%%%%%%%%%%%%%%%%%%%%%%%%%%%%%%%%%%
% You can use    "hebfont":                   %
%   IvriTeX David CLM glyphs as default font  %
% Or you can use "culmus":                    %
%   Nikud  Frank Ruehl gluphs as default font %
%   with following restrictions:              %
%     - You can not use utf8 encoding         %
%     - You must use latex->dvips->ps2pdf     %
%       sequence in order to get output file  %
%       You also must use ghostscript 7.07    %
%       later versions fail to convert texts  %
%                                             %
%      You can not use them together          %
%                                             %
%%%%%%%%%%%%%%%%%%%%%%%%%%%%%%%%%%%%%%%%%%%%%%%


%%%%%%%%%%%%%%%%%%%%%%%%%%%%%%%%%%%%%%%%%%%%%
% For IvriTeX you need following definition %
%%%%%%%%%%%%%%%%%%%%%%%%%%%%%%%%%%%%%%%%%%%%%

\usepackage{hebfont}

%%%%%%%%%%%%%%%%%%%%%%%%%%%%%%%%%%%%%%%%%%%%
% For Nikud you need following definitions %
% They are commented out by default.       %
%%%%%%%%%%%%%%%%%%%%%%%%%%%%%%%%%%%%%%%%%%%%
%
% \usepackage{culmus}
%
% \HeblatexRedefineL
% %  This is requred to enable \L{...}
% %  sequence
%
%%%%%%%%%%%%%%%%%%%%%%%%%%%%%%%%%%%%%%%%%%%%

%%%%%%%%%%%%%%%%%%%%%%%%%%%%%%%%
% English fonts in hebrew text %
%%%%%%%%%%%%%%%%%%%%%%%%%%%%%%%%

\usepackage{newcent}

%%%%%%%%%%%%%%%%%%%%%%%%%%%%%%%%%%%%%%%%%%%%
% This forces LaTeX use bolder fonts for   %
% English that look better together with   %
% Hebrew. Also it is recommended to use it %
% with Nikud package                       %
%%%%%%%%%%%%%%%%%%%%%%%%%%%%%%%%%%%%%%%%%%%%

%%%%%%%%%%%%%%%%%%%%%%%%%%%%%%%%%%%%%
% LaTeX do not support hyphenation 
% for Hebrew so it is good idea to use 
% this definition to prevent text
% outside the normal layout
%%%%%%%%%%%%%%%%%%%%%%%%%%%%%%%%%%%%%%

\global\emergencystretch = .8\hsize

%%%%%%%%%%%%%%%%%%%%%%%%%%%%%%%%%%%%%%
% Now we start document in Hebrew 
%%%%%%%%%%%%%%%%%%%%%%%%%%%%%%%%%%%%%%


%BIDION
\title{שימוש ב-BiDiTeX}
\author{ארתיום בייליס (טונקיך)}
\begin{document}
\maketitle
\section{דוגמאות לשימוש.}
עכשיו אפשר לכתוב טקסט בעברית שמשולב עם אנגלית בצורה נורמלית: "Some english" ועכשיו מספרים 1, 2, 3. ועכשיו סוגרים (עגולים) [מרובעים] או \{מסולסלים\} או אפילו סימני "<>". הכל עובד. כמו כן, אפשר לשלב תגים שונים כמו \emph{הדגשה} וזה לא יפריע או אפשר אפילו לשלב בצורה כזהו: "Long english \emph{text} with italic".

%BIDILTR
\unsethebrew
\section{English - אנגלית.}

You can also write in English and insert some Hebrew inside, for example -- you can insert a word ``שלום" or any other text and it will be treated correctly. However in this case you should use tag \%BIDILTR in order to make it work.

\sethebrew
%BIDION
\section{נוסחאות.}
כמו כן, ניתן לשלב נוסחאות קצרות בתוך הטקסט כמו $e=mc^2$ וזה יטופל באופן נכון. אבל אם רוצים לכתוב נוסחה מסובכת במספר שורות או כל דבר מסובך אחר יש לבטל את מנגנון הכיווניות:

%BIDIOFF
\begin{equation} \label{eq:myequ}
\forall{x}.\forall{y}.f(x+y)=f(x) \cdot f(y)  \to 
f(x)=e^x=\sum^{\infty}_{n=0}{\frac{x^n}{n!}}
\end{equation}
%BIDION
ולאחר מכן ניתן להחזיר אותו מחדש.

יש לשים לב שבד"כ כל התוכן של פקודות מטופל\footnote{הכוונה לתוכן שנמצא בין הסוגריים -- הפרמטרים של הפקודה} -- פרט לפקודות מיוחדות כמו begin, end, L, R. במידה ורוצים לגרום ל-biditex להתעלם מתוכן פקודות מסיימות -- לדגומה תוכן של תג ref, אז צריך לסמן זאת ל-biditex ע"י תג BIDITAG\%.


%BIDITAG ref

אחרי הגדרת התג ניתן להתייחס לנוסחה (\ref{eq:myequ}) שדיברנו עליה

\end{document}
%BIDIOFF
